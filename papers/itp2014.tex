\documentclass{llncs}
\usepackage{amsfonts}

\usepackage{holtexbasic}
\usepackage{alltt}

\title{Mechanising Theoretical Bounds in Planning Problems}
\author{Mohammad Abdulaziz \and Charles Gretton \and Michael Norrish}

\institute{Canberra Research Lab., NICTA}

\begin{document}
\maketitle

\begin{abstract}
We verify a theorem in a recent IJCAI paper on planning bounds, finding and fixing a significant bug along the way.
\end{abstract}

\section{Introduction}
\label{sec:introduction}

\section{Modelling and Definitions}
\label{sec:modelling}

The

legal action sequence


Another important concept is the concept of a legal state. A state \(s\)is legal in a specific planning problem if its domain is a subset
 of the domain of the initial state of the problem \(I_\pi\), formally stated as follows:
\begin{equation}
\label{eqn:legalstate}
FDOM (s) = FDOM (I_\pi)
\end{equation}


Another important function is the function \begin{math} L :\alpha \ problem \rightarrow \mathbb{N} \end{math}
\begin{equation}
\label{eqn:L}
L(\pi) =
\end{equation}
that takes as an argument a planning problem \( \pi \) and returns back the longest action sequence possible to
 reach any legal state \(s_1\) from any other legal state \(s_0\) in the
problem \(\pi\).

\begin{alltt}
\HOLTokenTurnstile{} \HOLConst{planning_problem} \HOLFreeVar{\ensuremath{\Pi}} \HOLTokenImp{} \HOLConst{\ensuremath{\ell}} \HOLFreeVar{\ensuremath{\Pi}} \HOLTokenLeq{} 2 \HOLTokenExp{} \ensuremath{|}\HOLConst{FDOM} \HOLFreeVar{\ensuremath{\Pi}}.I\ensuremath{|}
\end{alltt}

Here is some text featuring \HOLinline{\HOLConst{stitch}} and \HOLinline{\HOLConst{\ensuremath{\ell}}}.

\section{A Bug}
\label{sec:bug}

One of the main contributions of our work, is that we have realised the presence of a bug in \cite{1}. In this section we will discuss
what that bug is. In \cite{1} one of the main contributions is a lemma stating the following


This lemma is not correct. The counter example for its validity is



Another problem is inherent in the definition of the function \( L\). The problem is that this definition
enumerates all the possible legal states and then enumerates optimal legal action sequences from each of them to the other
by taking the shortest of them by applying the function \( min\) to the set of all paths. The problem is when
there is not a path between two legal states. In this case the function \(min\) has no defined output.

We have fixed that by redefining \(L(\pi)\) in the following way.

\begin{equation}
\label{eqn:Lnew}
L(\pi) =
\end{equation}


In this definition we have resolved the issues.
Using this definition, the lemma \ref{label:lemma1} is valid.



\section{The Theorem}
\label{sec:theorem}

\subsection{Proof Strategy}
\label{sec:proof-strategy}

\subsection{The Proof and Its Lemmas}
\label{sec:lemmas}

\section{Conclusion}

\end{document}
